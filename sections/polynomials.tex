\documentclass[../main.tex]{subfiles}

\begin{exercise}(\cite{alfutova}, 6.82)
    Factorize the following into factors with real-valued coefficients:
    \begin{enumerate}
        \item $x^4 + 4$;
        \item $2x^3 + x^2 + x - 1$;
        \item $(a + b + c)^3 - a^3 - b^3 - c^3$;        
        \item $x^{10} + x^5 + 1$;
        \item $a^3 + b^3 + c^3 - 3abc$;
        \item $x^3 + 3xy + y^3 - 1$;
        \item $x^2y^2 - x^2 - y^2 + 4xy + 1$;
        \item $(x - y)^5 + (y - z)^5 + (z - x)^5$;
        \item $a^8 + a^6b^2 + a^4b^4 + a^2b^6 + b^8$;
        \item $(x^2 + x + 1)^2 + 3x(x^2 + x + 1) + 2x^2$;
        \item $a^4 + b^4 + c^4 - 2a^2b^2 - 2b^2c^2 - 2a^2c^2$;
        \item $(x+1)(x+3)(x+5)(x+7) + 15$.
    \end{enumerate}
\end{exercise}
\begin{solution}
    \begin{enumerate}
        \item $x^4 + 4 = (x^2 + 2x + 2)(x^2 - 2x + 2)$. Since neither of the quadratic polynomials have real roots, further decomposition is impossible.
        \item $2x^3 + x^2 + x - 1 = 2x^3 - x^2 + 2x^2 - x + 2x - 1 = (2x-1)(x^2 + x + 1)$. Again, the quadratic factor does not have real roots, so the decomposition is final.
        \item \begin{align*}
            (a + b + c)^3 - a^3 - b^3 - c^3 &= ((a + b + c)^3 - a^3) - (b^3 + c^3) \\
            &= (b + c)((a+b+c)^2 + (a+b+c)a + a^2) \\
            &\hspace{10mm}- (b+c)(b^2 - bc + c^2) \\
            &= (b+c)(a^2 + b^2 + c^2 + 2ab + 2bc + 2ac \\
            &\hspace{10mm}+ a^2 + ab + ac + a^2 - b^2 + bc - c^2) \\
            &= (b+c)(3a^2 + 3ab + 3bc + 3ac) \\
            &= 3(b+c)(a(a+b) + c(a+b)) \\
            &= 3(a+b)(b+c)(a+c).
        \end{align*}        
        \item \begin{align*}
            x^{10} + x^5 + 1 &= x^{10} - x^7 + x^7 - x^4 + x^5 - x^2 + x^4 - x + x^2 + x + 1 \\
            &= x^7(x^3 - 1) + x^4(x^3 - 1) + x^2(x^3 - 1) + x(x^3 - 1) + x^2 + x + 1 \\
            &= (x^7 + x^4 + x^2 + x)(x^3 - 1) + x^2 + x + 1 \\
            &= (x^2 + x + 1)((x^7 + x^4 + x^2 + x)(x - 1) + 1) \\
            &= (x^2 + x + 1)(x^8 - x^7 + x^5 - x^4 + x^3 - x + 1).
        \end{align*}
        \item \begin{align*}
            a^3 &+ b^3 + c^3 - 3abc = (a^3 - ab^2) + (ab^2 - abc) \\
            &\hspace{10mm}+ (b^3 - bc^2) + (bc^2 - abc) + (c^3 - ca^2) + (ca^2 - abc) \\
            &= a(a^2 - b^2) + ab(b - c) + b(b^2 - c^2) \\
            &\hspace{10mm}+ bc(c - a) + c(c^2 - a^2) + ac(a - b) \\
            &= a((a^2 - b^2) + c(a - b)) + b((b^2 - c^2) + a(b - c)) \\
            &\hspace{10mm}+ c((c^2 - a^2) + b(c - a)) \\
            &= a(a - b)(a + b + c) + b(b - c)(a + b + c) + c(c - a)(a + b + c) \\
            &= (a + b + c)(a^2 - ab + b^2 - bc + c^2 - ac).
        \end{align*}
        \item Note that this is the same as the previous polynomial, if we make a substitution $a = x, b = y, c = -1$. Thus, $x^3 + y^3 - 1 + 3xy = (x + y - 1)(x^2 - xy + y^2 + y + x + 1)$.
        \item $x^2y^2 - x^2 - y^2 + 4xy + 1 = x^2y^2 + 2xy + 1 - (x^2 - 2xy + y^2) = (xy + 1)^2 - (x - y)^2 = (xy - x + y + 1)(xy + x - y + 1)$.        
        \item \begin{align*}
            (x - y)^5 &+ (y - z)^5 + (z - x)^5 = ((x - z) + (z - y))^5 - (z - y)^5 - (x - z)^5 \\
            &= (x - z)^5 + 5(x - z)^4(z - y) + 10(x - z)^3(z - y)^2 + 10(x - z)^2(z - y)^3 \\
            &\hspace{10mm}+ 5(x - z)(z - y)^4 + (z - y)^5 - (z - y)^5 - (x - z)^5 \\
            &= 5(x - z)(z - y)((x - z)^3 + 2(x - z)^2(z - y) + 2(x - z)(z - y)^2 + (z - y)^3) \\
            &= 5(x - z)(z - y)((x - z)^2((x - z) + (z - y)) + (z - y)^2((x - z) + (z - y)) \\
            &\hspace{10mm}+ (x - z)(z - y)(x - z + z - y)) \\
            &= 5(x - z)(z - y)((x - z)^2(x - y) + (z - y)^2(x - y) + (x - z)(z - y)(x - y)) \\
            &= 5(x - z)(z - y)(x - y)((x - z)^2 + (z - y)^2 + (x - z)(z - y)) \\
            &= 5(x - y)(y - z)(z - x)(x^2 - 2xz + z^2 + z^2 - 2zy + y^2 + xz - xy - z^2 + zy) \\
            &= 5(x - y)(y - z)(z - x)(x^2 + y^2 + z^2 - xy - yz - xz).
        \end{align*}
        \item \begin{align*}
            a^8 + a^6b^2 &+ a^4b^4 + a^2b^6 + b^8 = \frac{(a^2 - b^2)(a^8 + a^6b^2 + a^4b^4 + a^2b^6 + b^8)}{a^2 - b^2} \\
            &= \frac{a^{10} - b^{10}}{a^2 - b^2} = \frac{(a^5 - b^5)(a^5 + b^5)}{(a-b)(a+b)} \\
            &= \frac{(a - b)(a^4 + a^3b + a^2b^2 + ab^3 + b^4)(a+b)(a^4 - a^3b + a^2b^2 - ab^3 + b^4)}{(a-b)(a+b)} \\
            &= (a^4 + a^3b + a^2b^2 + ab^3 + b^4)(a^4 - a^3b + a^2b^2 - ab^3 + b^4)
        \end{align*}
        \item $(x^2 + x + 1)^2 + 3x(x^2 + x + 1) + 2x^2 = (x^2 + x + 1)((x^2 + x + 1) + x) + 2x((x^2 + x + 1) + x) = (x^2 + x + 1)(x + 1)^2 + 2x(x + 1)^2 = (x + 1)^2(x^2 + 3x + 1)$. The decomposition is final since $x^2 + 3x + 1$ does not have real roots.
        \item \begin{align*}
            a^4 + b^4 + c^4 &- 2a^2b^2 - 2b^2c^2 - 2a^2c^2 = a^4 + b^4 + c^4 - 2a^2b^2 + 2b^2c^2 - 2a^2c^2 - 4b^2c^2 \\
            &= (a^2 - b^2 - c^2)^2 - 4b^2c^2 = (a^2 - b^2 - c^2 - 2bc)(a^2 - b^2 - c^2 + 2bc) \\
            &= (a^2 - (b + c)^2)(a^2 - (b - c)^2)) \\
            &= (a + b + c)(a - b - c)(a - b + c)(a + b - c).
        \end{align*}
        \item \begin{align*}
            (x+1)(x+3)(x+5)(x+7) + 15 &= x^4 + 16x^3 + 86x^2 + 176x + 120 \\
            &= (x^4 + 2x^3) + (14x^3 + 28x^2) + (58x^2 + 116x) + (60x + 120) \\
            &= (x+2)(x^3 + 14x^2 + 58x + 60) \\
            &= (x+2)((x^3 + 6x^2) + (8x^2 + 48x) + (10x + 60)) \\
            &= (x+2)(x+6)(x^2 + 8x + 10).
        \end{align*}
        The decomposition is final since the quadratic does not have real roots.
    \end{enumerate}
\end{solution}

\begin{exercise}(\cite{alfutova}, 6.83)
    Prove that $x^4 + px^2 + q$ can always be represented as a product of two quadratic polynomials.
\end{exercise}
\begin{solution}
    Consider two cases: i) $p^2 - 4q \geq 0$. In this case, we can make a substitution $y = x^2$ which yields quadratic equation $y^2 + py + q$ which has real roots, $y_1, y_2$. Then $x^4 + px^2 + q = (y - y_1)(y - y_2) = (x^2 - y_1)(x^2 - y_2)$, as required.

    ii) $p^2 - 4q < 0$. In this case note that $q > 0$ and $2\sqrt{q} > p$. Then it is easy to check that $(x^2 + x\sqrt{2\sqrt{q} - p} + \sqrt{q})(x^2 - x\sqrt{2\sqrt{q} - p} + \sqrt{q}) = x^4 + px^2 + q$.
\end{solution}

\begin{exercise}(\cite{alfutova}, 6.85)
    Simplify the expression $\frac{(a + b + c)^5 - a^5 - b^5 - c^5}{(a + b + c)^3 - a^3 - b^3 - c^3}$
\end{exercise}
\begin{solution}
    TODO!!
\end{solution}

\begin{exercise}(\cite{alfutova}, 6.89)
    Prove that if $a + b + c = 0$, then $2(a^5 + b^5 + c^5) = 5abc(a^2 + b^2 + c^2)$.
\end{exercise}
\begin{solution}
    TODO!!!
\end{solution}

\begin{exercise}(\cite{alfutova}, 6.91)
    Prove that $\sqrt{17}$ is not a rational number.    
\end{exercise}
\begin{solution}
    Consider the polynomial $x^2 - 17$ that has $\sqrt{17}$ as a root. According to the theorem about rational roots of polynomials with integer coefficients, if $p/q$ is a root in the lowest terms of a polynomial $a_nx^n + ... + a_1x + a_0$ with integer coefficients, then $q \mid a_n$ and $p \mid a_0$. Applying this theorem to $x^2 - 17$, if $\sqrt{17} = p/q$, with $(p, q) = 1$, then $p \mid -17$ and $q \mid 1$. This implies the following possible values: $p = \pm 1, \pm 17$ and $q = \pm 1$. But then $p/q = \pm 1, \pm 17$, and none of these combinations could equal $\sqrt{17}$, a contradiction. Hence, $\sqrt{17}$ could not be represented as a rational number.
\end{solution}

\begin{exercise}(\cite{alfutova}, 6.92)
    Prove that $\cos 20$ is an irrational number.
\end{exercise}
\begin{solution}
    Recall that $\cos 3x = 4\cos^3 x - 3 \cos x$. From this, $1/2 = \cos 60 = 4\cos^3 20 - 3\cos 20$, or $8\cos^3 20 - 6 \cos 20 - 1 = 0$. This means that $\cos 20$ is a root of the polynomial $8x^3 - 6x - 1$. According to the theorem about rational roots of polynomials with integer coefficients, if $\cos 20 = p/q$, where $p/q$ is in the lowest terms, then $p \mid -1, q \mid 8$. That is, $\cos 20 = p/q = \pm 1, \pm 1/2, \pm 1/4, \pm 1/8$. Now, we can check by direct substitution that neither of these numbers is a root of the polynomial in question. Hence, $\cos 20$ cannot be equal any of them, which is a contradiction. Thus, $\cos 20$ must be an irrational number.
\end{solution}

\begin{exercise}(\cite{alfutova}, 6.94)
Solve equations:

a) $x^4 + x^3 - 3a^2x^2 - 2a^2x + 2a^4 = 0$;

b) $x^3 - 3x = a^3 + a^{-3}$.
\end{exercise}
\begin{solution}
    TODO!!!
\end{solution}

\begin{exercise}(\cite{alfutova}, 6.98)
 Prove that the polynomial $P(x) = 1 + x + \frac{x^2}{2!} + ... + \frac{x^n}{n!}$ does not have roots with multiplicity more than one.
\end{exercise}
\begin{solution}
    Note that if $a$ is a root of $P(x)$ with multiplicity $k$, i.e., $P(x) = (x - a)^kQ(x)$, then $a$ is a root of $P'(x)$ with multiplicity $k-1$. So, if $P(x)$ has a root $a$ with multiplicity $k > 1$, then $\frac{x^n}{n!} = P(x) - P'(x) = (x - a)^{k-1}T(x)$. This implies that $a$ must be a root of $\frac{x^n}{n!}$. However, the latter has only one root, $0$, which is not a root of $P(x)$. A contradiction. Hence, $P(x)$ cannot have roots of multiplicity $>1$.
\end{solution}

\begin{exercise}(\cite{alfutova}, 6.101)
Prove that a polynomial $P(x)$ is divisible by its derivative iff $P(x) = a(x - x_0)^n$, for some $a, x_0, n$.
\end{exercise}
\begin{solution}
$\Leftarrow$: If $P(x) = a(x - x_0)^n$, then $P'(x) = an(x - x_0)^{n-1}$ and obviously $P'(x)$ divides $P(x)$.

$\Rightarrow$: Let $P'(x)$ divide $P(x)$, i.e., let $P(x) = P'(x)T(x)$. Then note that $\delta P'(x) = \delta P(x) - 1$, which implies that $\delta T(x) = 1$, i.e., $T(x) = a(x - b)$. Then $P'(x) = (P'(x)T(x))' = P''(x)T(x) + P'(x)T'(x) = P''(x)T(x) + aP'(x)$. Rearranging gives $P'(x)(1 - a) = P''(x)T(x)$. Now either $a = 1$, and so $P''(x)T(x) = 0$, which implies that $P''(x) = 0$ and $P'(x) = c$, or $P'(x) = \frac{P''(x)T(x)}{1-a}$, i.e., $P'(x)$ is divisible by $P''(x)$. In the latter case we can apply the same argument to $P'(x)$ to show that either $P''(x) = c$ or $P''(x)$ is divisible by $P'''(x)$. Now consider the sequence of consecutive derivatives, $P_0(x) = P(x), P_1(x) = P'(x), ..., P_{n+1}(x) = P_n'(x)$. By extending previous argument using induction, at any $k$, either $P_{k+1}(x)$ divides $P_k(x)$, or $P_k(x) = c$ and $P_{k+1}(x) = P_{k+2}(x) = ... = 0$. Note that if $\delta P_k(x) > 0$, then $\delta P_{k+1}(x) = \delta P_k(x) - 1$. Hence, $P_{n+1}(x) = P^{(n+1)}(x) = 0$, and $P_n(x) = c$. Now, if $Q'(x) = c_1(x - a)^m$ for some $c_1, a, m$, then $Q(x) = c_2(x - a)^{m+1}$. Since $P_n(x) = c = c_0(x - a)^0$, then $P_{n-1}(x) = c_1(x - a)^1$. By induction we get that $P(x) = P_0(x) = P_{n-n}(x) = c_n(x - a)^n$, i.e., $P(x)$ has the required form.
\end{solution}

\begin{exercise}(\cite{alfutova}, 6.105)
    Prove that $P(x) = a_0 + a_1x + ... + a_nx^n$ has $-1$ as a root of multiplicity $m$ iff
    \[
    \left\{ 
    \begin{array}{l}
        a_0 - a_1 + a_2 - ... + (-1)^na_n = 0, \\
        -a_1 + 2a_2 - 3a_3 + ... + (-1)^n na_n = 0, \\
        ... \\
        -a_1 + 2^ma_2 - 3^ma_3 + ... + (-1)^n n^m a_n = 0.
    \end{array}
    \right.
    \]
\end{exercise}
\begin{solution}
    TODO!!!
\end{solution}

\begin{exercise}(\cite{alfutova}, 6.106)
    Prove that the polynomial $P(x) = (x^{n+1} - 1)(x^{n+2} - 1)...(x^{n+m} - 1)$ is divisible by $Q(x) = (x^1 - 1)(x^2 - 1)...(x^m - 1)$.
\end{exercise}
\begin{solution}
    TODO!!!
\end{solution}

\begin{exercise}(\cite{alfutova}, 6.109)
    Let 
    \[\left\{
    \begin{array}{l}
        x + y + z = a, \\
        \frac{1}{x} + \frac{1}{y} + \frac{1}{z} = \frac{1}{a}. 
    \end{array}
    \right.\]
Prove that at least one of $x, y, z$ must equal $a$.
\end{exercise}
\begin{solution}
    The result can be obtained from decomposing as follows $(a - x)(a - y)(a - z) = a^3 - a^2z - a^2y + ayz - a^2x + axz + axy - xyz = a^3 - a^2(x + y + z) + a(xy + yz + xz) - xyz = a^3 - a^2\cdot a + a \frac{xyz}{a} - xyz = 0$.
\end{solution}

\begin{exercise}(\cite{alfutova}, 6.112)
    Construct a cubic polynomial whose roots are squares of the roots of the polynomial $x^3 + x^2 - 2x - 1 = 0$.
\end{exercise}
\begin{solution}
    Let $x_1, x_2, x_3$ be the roots of $x^3 + x^2 - 2x - 1$. Then the required polynomial has the form $(x - x_1^2)(x - x_2^2)(x - x_3^2) = x^3 - x^2x_3^2 - x^2x_2^2 + xx_2^2x_3^2 - x^2x_1^2 + xx_1^2x_3^2 + xx_1^2x_2^2 - x_1^2x_2^2x_3^2 = x^3 - (x_1^2 + x_2^2 + x_3^2)x^2 + (x_1^2x_2^2 + x_1^2x_3^2 + x_2^2x_3^2)x - x_1^2x_2^2x_3^2$.

    Note that the Vieta's theorem applied to the roots of the original polynomial gives $x_1 + x_2 + x_3 = -1, x_1x_2 + x_2x_3 + x_1x_3 = -2, x_1x_2x_3 = 1$. These relations give $x^3 - (x_1^2 + x_2^2 + x_3^2)x^2 + (x_1^2x_2^2 + x_1^2x_3^2 + x_2^2x_3^2)x - x_1^2x_2^2x_3^2 = x^3 - ((x_1 + x_2 + x_3)^2 - 2(x_1x_2 + x_2x_3 + x_1x_3))x^2 + ((x_1x_2 + x_2x_3 + x_1x_3)^2 - 2x_1x_2x_3(x_1 + x_2 + x_3))x - (x_1x_2x_3)^2 = x^3 - (1 - 2(-2))x^2 + (4 - 2(-1))x - 1 = x^3 - 5x^2 + 6x - 1$.
\end{solution}

\begin{exercise}(\cite{alfutova}, 6.115)
    Assume that all the roots of the equation $x^3 + px^2 + qx + r = 0$ are positive. What condition on $p, q$ and $r$ is needed to ensure that segments with lengths equal to these roots can form a triangle?
\end{exercise}
\begin{solution}
    TODO!!!
\end{solution}

\begin{exercise}(\cite{alfutova}, 6.116b)
    Assume that
    \[
        \begin{array}{c}
            x + y + z = u + v + t, \\
            x^2 + y^2 + z^2 = u^2 + v^2 + t^2, \\
            x^3 + y^3 + z^3 = u^3 + v^3 + t^3.
        \end{array}
    \]
    Prove that for every natural number $n$ we also have $x^n + y^n + z^n = u^n + v^n + t^n$.
\end{exercise}
\begin{solution}
    TODO!!!
\end{solution}

\begin{exercise}(\cite{alfutova}, 6.123)
    Solve in positive integers the system
    \[
    \left\{
    \begin{array}{l}
         x + y = uv, \\
         u + v = xy.
    \end{array}
    \right.
    \]    
\end{exercise}
\begin{solution}
\textbf{First solution.}

    Consider three scenarios:

    1) $x + y = xy$: this implies $x = y(x - 1)$. Since $x$ and $x-1$ are mutually prime, this is possible only if $x = 2$. This yields $y = x = 2$, and $uv = 4, u + v = 4$, i.e., $x = y = u = v = 2$. Thus, one solution is $(x, y, u, v) = (2, 2, 2, 2)$.

    2) $x + y > xy$: note that this is only possible if either $x = 1$ or $y = 1$ or $x = y = 2$, in any other case $xy \geq x + y$. The latter of the three sub-cases lead to $x = y = u = v = 2$. Now, if $x = 1$, we have $1 + y = uv, u + v = y$, which yields $u(v-1) = v+1$ (by substituting for $y$ and rearranging), or $u = \frac{v+1}{v-1}$. Note that $\frac{n+2}{n}$ can only be an integer if either $n = 1$ or $n = 2$. This implies that $v = 2$ or $v = 3$, which corresponds to $u = 3$ and $u = 2$, respectively. Both cases lead to $y = 5$. Finally, in the last sub-case when $y = 1$, by the symmetric argument we obtain that $x = 5$ and $u = 2, v = 3$ or $u = 3, v = 2$. Thus we obtain four more solutions: $(1, 5, 2, 3), (1, 5, 3, 2), (5, 1, 2, 3), (5, 1, 3, 2)$.

    3) $x + y < xy$: note that this is equivalent to $uv < u + v$, and so due to symmetry we can apply the previous argument to $u, v$ instead of $x, y$. This gives another four solutions which are permutations of the previous four: $(2, 3, 1, 5), (3, 2, 1, 5), (2, 3, 5, 1), (3, 2, 5, 1)$.

    Thus, the given equation has 9 solutions in positive integers.

\textbf{Second Solution.}
TODO!!!
\end{solution}

\begin{exercise}(\cite{alfutova}, 6.129)
Let $x_1 < x_2 < ... < x_n$ be real numbers. Prove that for any real $y_1, ..., y_n$ there is a unique polynomial $f(x)$, of degree at most $n-1$, such that $f(x_1) = y_1, ..., f(x_n) = y_n$.
\end{exercise}
\begin{solution}
    Define $f_i(x) = \frac{\prod_{1 \leq j \leq n, j \not= i} (x - x_j)}{\prod_{1 \leq j \leq n, j \not= i} (x_i - x_j)} = \frac{(x - x_1)...(x - x_{i-1})(x - x_{i+1})...(x - x_n)}{(x_i - x_1)...(x_i - x_{i-1})(x_i - x_{i+1})...(x_i - x_n)}$. It is straightforward to see that $f_i(x_i) = 1$ and $f_i(x_j) = 0, \forall j \not= i$. Note that each $f_i(x)$ is a polynomial of degree at most $n - 1$ since each is a product of $n-1$ monics. Now, let $f(x) = \sum_{1 \leq i \leq n} y_if_i(x)$. Again, it is clear that $f(x_i) = y_i, \forall 1 \leq i \leq n$. And it is also clear that $f(x)$ is a sum of polynomials of degree at most $n-1$, and hence itself has degree at most $n-1$. This concludes the existence part of the proof. 
    
    To show uniqueness, let $g(x)$ be another polynomial of degree at most $n-1$ such that $g(x_i) = y_i, \forall i$. Then $(f - g)(x) = f(x) - g(x)$ is a polynomial of degree at most $n-1$ with roots $x_1, ..., x_n$ (since $f(x_i) - g(x_i) = y_i - y_i = 0, \forall i$). But non-zero polynomial of degree at most $n-1$ cannot have $n$ distinct roots, hence $f(x) - g(x) = 0$, or $g(x) = f(x)$, as required.
\end{solution}

\begin{exercise}(\cite{alfutova}, 6.130)
    Let $A, B, C$ be the remainders of dividing $P(x)$ by $(x - a), (x - b)$ and $(x - c)$. Find the remainder of dividing $P(x)$ by the product $(x - a)(x - b)(x - c)$.
\end{exercise}
\begin{solution}
    TODO!!!
\end{solution}

\begin{exercise}
a) (\cite{alfutova}, 6.136) Let a, b, c be three distinct numbers. Solve the system

\[
\left\{
\begin{array}{l}
     z + ay + a^2x + a^3 = 0, \\
     z + by + b^2x + b^3 = 0, \\
     z + cy + c^2x + c^3 = 0.
\end{array}
\right.
\]

b) (\cite{alfutova}, 6.137) Let a, b, c be three distinct numbers. Prove that the system of equalities
\[
\left\{
\begin{array}{l}
     x + ay + a^2z = 0, \\
     x + by + b^2z = 0, \\
     x + cy + c^2z = 0,
\end{array}
\right.
\]
implies that $x = y = z = 0$.
\end{exercise}
\begin{solution}
a) Note that $a, b, c$ are the three roots of the polynomial $p(t) = t^3 + xt^2 + yt + z$. Now, Vieta's formulas give us $x = -(a + b + c), y = ab + bc + ac, z = -abc$.

b) Note that these equalities imply that $a, b, c$ are three roots of the quadratic polynomial $p(t) = zt^2 + yt + x$. But non-degenerate quadratic cannot have three roots. Hence, the polynomial must be identically zero, i.e., $x = y = z = 0$, as required.
\end{solution}

\begin{exercise}(\cite{alfutova}, 6.138)
It is known for $f(x) = x^{10} + a_9x^9 + ... + a_1x + a_0$ that $f(1) = f(-1), f(2) = f(-2), ..., f(5) = f(-5)$. Prove that $f(x) = f(-x)$ for all real $x$.   
\end{exercise}
\begin{solution}
    TODO!!!
\end{solution}

\begin{exercise}(\cite{alfutova}, 6.141)
    Solve the system
\[
\left\{
\begin{array}{l}
     \frac{x_1}{a_1 - b_1} + \frac{x_2}{a_1 - b_2} + ... + \frac{x_n}{a_1 - b_n} = 1, \\
     \frac{x_1}{a_2 - b_1} + \frac{x_2}{a_2 - b_2} + ... + \frac{x_n}{a_2 - b_n} = 1, \\
     .................... \\
     \frac{x_1}{a_n - b_1} + \frac{x_2}{a_n - b_2} + ... + \frac{x_n}{a_n - b_n} = 1.
\end{array}
\right.
\]
\end{exercise}
\begin{solution}
    TODO!!!
\end{solution}
